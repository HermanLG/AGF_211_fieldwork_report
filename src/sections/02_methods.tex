\section{Instruments, Data and Methods} \label{sec:methods}

\subsection{Darcy's law}

Darcy's law describes the hydraulic flux of a fluid through a cross section using the hydraulic conductivity.\\
Without gravitational influence the instantaneous flux $q[m/s]$ though a porous medium with cross section $A[m^2]$ and permeability $k[m^2]$ of a fluid with viscosity $\mu[Pa \; s]$ is given by equation \ref{darcys law}

\begin{equation}\label{darcys law}
    q = -\frac{k}{\mu}\nabla p
\end{equation}

Where $\nabla p[Pa]$ is the pressure difference over length $L[m]$. The negative sign means that the fluid flows from high pressure area to a low pressure area. 
In the case of vertical flux and by assuming a static fluid pressure we can relate the pressure to the height of the fluid by Stevin's law.

\begin{equation}\label{darcys law diff eq}
    q(t) = -\frac{\rho g k}{\mu L}\Delta h(t) \\
\end{equation}

Where $g$ is the gravitational acceleration, and $\rho$ is the density of the fluid. Note that we have added a time dependency of the height and thus flux. As the volume is filled up, the height(and pressure) increase, altering the flux. \\ 
The solution to the differential equation \ref{darcys law diff eq} shows an exponential relation between height and time.  

\begin{equation}\label{diff eq solution}
    h(t) = h_0 \: exp\left(-\frac{\rho g k t}{\mu L}\right)
\end{equation}

Linearizing equation \ref{diff eq solution} and solving for the permeability $k$ yields:

\begin{equation}\label{lin.diff sol}
    k = \frac{a \rho g}{\mu L}
\end{equation}

Were as $a$ is a linear relation coefficient, which can be found by regression. 

\subsection{Empirical model}




