\section{Instruments, Data and Methods} \label{sec:methods}

\subsection{Darcy's law}

Darcy's law describes the hydraulic flux of a fluid through a cross section using the hydraulic conductivity.\\
Without gravitational influence the instantaneous flux $q[m/s]$ though a porous medium with cross section $A[m^2]$ and permeability $k[m^2]$ of a fluid with viscosity $\mu[Pa \; s]$ is given by equation \ref{darcys law}

\begin{equation}\label{darcys law}
    q = -\frac{k}{\mu}\nabla p
\end{equation}

Where $\nabla p[Pa]$ is the pressure gradiant over the volume.
The negative sign means that the fluid flows from high pressure area to a low pressure area.
In the case of vertical flux and by assuming a hydrostatic pressure we can relate the pressure to the height of the fluid by Stevin's law.
We then consider the seawater to be incompressible and the height of the hole to be negligible compared to the radius of the earth, so that $g$ and $\rho$ can be assumed constant.

\begin{equation}\label{darcyslawdiffeq}
    q(t) = -\frac{\rho g k}{\mu L}\Delta h(t) \\
\end{equation}

Where $g$ is the gravitational acceleration, $\rho$ is the density of the fluid and $L[m]$ is the distance the fluid percolates.
Note that we have added a time dependency of the height and thus flux.
Darcy's law does not account for a change in pressure, but as our holes fills with water the height and pressure changes, altering the flux.
Only when the pressure gradient is fairly large we are able to compute the permeability, when the gradient goes towards zero the equation is no longer valid.  \\
We assume the other parameters to remain constants.
The solution to the first-order linear ordinary differential equation \ref{darcyslawdiffeq} is given on the form:

\begin{equation}
    y(t) = A \: \frac{d}{dt}\left( y(t) \right) \Rightarrow y(t) = c_1 \: e^{\frac{t}{A}}
\end{equation}

The solution shows an exponential relation between height and time.

\begin{equation}\label{diff eq solution}
    h(t) = h_0 \: exp\left(-\frac{\rho g k t}{\mu L}\right)
\end{equation}

Linearizing and solving for the permeability $k$ yields:

\begin{equation}\label{lin.diff sol}
    k = -a\frac{\mu L}{\rho g}
\end{equation}

Were as $a$ is a linear relation coefficient defined as $a = \log\left(\frac{h(t)}{h_0}\right)t^{-1}$ which can be found by regression.

\subsection{Empirical model}\label{subsec:empirical-model}
By measuring the salinity and temperature of young sea ice, one can use an empirical approximation to estimate the permeability of the ice.
This section will follow the reasoning by \cite{src/citations/citation-267564215.bib} and references within.

\begin{equation}\label{emp.eq.perm}
    \Pi = 10^{-17}[10^3(1 - \phi_v)]^{3.1}
\end{equation}

$\Pi$ is the permeability, whereas $\phi_v$ is the approximated solid volume fraction, given in equation \ref{solid volume fraction}

\begin{equation}\label{solid volume fraction}
    \phi_v = 1 - \frac{S_{bu}}{S_{br}}
\end{equation}

$S_{bu}$ is the bulk salinity of the ice, which are found through the measurements of the ice core samples.
From~\cite{src/citations/pericles_21562202c114.bib}, we can estimate the brine salinity based on the temperature[$C^{\circ}$] of the ice.

\begin{equation}\label{brine salinity from temp}
    S_{br} = -21.4T - 0.886T^2 - 0.0170T^3
\end{equation}


\subsection{Experimental protocol}\label{subsec:experimental-protocol}
We cut 6 square holes into the sea-ice, hereby named station 1--6.
Station 1 was placed the closes to the ice edge, whiles station 6 the farthest into the fjord, see figure MAP OVER THE STATIONS.
This was done to analyse the ice structure for different ice depths, to see the variation in the sea-ice.
The sea ice was mostly covered in snow, with depth varying from 1cm to 25cm.
This was shoveled off, along with some slush ice.
We tried to keep at least one shallow layer of slush ice for each observation.
About a 30x30cm ice block was removed and laid on its side.
The block was heavily photographed, and the apparent structures were noted.
The block was then coated in PURPLE STUFF, to be abel to visualises the brine channels more clearly.
Both horizontal and vertical tests were done and analysed.
Fresh water ice has fewer brine channels than sea-water ice, and the PURPLE STUFF will not percolate as easily through fresh-water ice as it would for sea-water ice.\\
The stations were placed in the proximity of the ice cores drilled by [CITE THE ICE CORE GROUP].
All the salinity and temperature profiles of the ice samples are gathered and analysed by [CITE THE ICE CORE GROUP].\\
By measuring the amount of $^{18}O$ in the ice samples, one can find out if the water originates from precipitation or groundwater.
By using this so called $\delta ^{18}O$ test, we can figure out if the apparent fresh water ice layer found in the sea-ice originates from precipitation or the nearby fresh-water sources.\\
We attempted to perform permeability tests by the use of blind holes.
This would be done by drilling to a certain depth in the sea-ice, measure how far the water rises and how fast the water pressure stabilizes.
This could then be related to Darcy's law and thus estimate the permeability.
This requires the ice to be relatively porous, for water to flow thought it.
Another method of estimating the permeability is by an empirical approximation.
Here we would use the salinity and temperature measurements gathered by [CITE ICE CORE GROUP].
The permability if the different layers in the ice could tell us more about that type of ice which has formed

\subsection{Instruments}\label{subsec:instruments}




