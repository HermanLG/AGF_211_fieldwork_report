\section{Instruments, Data and Methods} \label{sec:methods}

\subsection{Darcy's law}

Darcy's law describes the hydraulic flux of a fluid through a cross section using the hydraulic conductivity.\\
Without gravitational influence the instantaneous flux $q[m/s]$ though a porous medium with cross section $A[m^2]$ and permeability $k[m^2]$ of a fluid with viscosity $\mu[Pa \; s]$ is given by equation \ref{darcys law}

\begin{equation}\label{darcys law}
    q = -\frac{k}{\mu}\nabla p
\end{equation}

Where $\nabla p[Pa]$ is the pressure gradiant over the volume.
The negative sign means that the fluid flows from high pressure area to a low pressure area.
In the case of vertical flux and by assuming a hydrostatic pressure we can relate the pressure to the height of the fluid by Stevin's law.
We then consider the seawater to be incompressible and the height of the hole to be negligible compared to the radius of the earth, so that $g$ and $\rho$ can be assumed constant.

\begin{equation}\label{darcyslawdiffeq}
    q(t) = -\frac{\rho g k}{\mu L}\Delta h(t) \\
\end{equation}

Where $g$ is the gravitational acceleration, $\rho$ is the density of the fluid and $L[m]$ is the distance the fluid percolates.
Note that we have added a time dependency of the height and thus flux.
Darcy's law does not account for a change in pressure, but as our holes fills with water the height and pressure changes, altering the flux.
Only when the pressure gradient is fairly large we are able to compute the permeability, when the gradient goes towards zero the equation is no longer valid.  \\
We assume the other parameters to remain constants.
The solution to the first-order linear ordinary differential equation \ref{darcyslawdiffeq} is given on the form:

\begin{equation}
    y(t) = A \: \frac{d}{dt}\left( y(t) \right) \Rightarrow y(t) = c_1 \: e^{\frac{t}{A}}
\end{equation}

The solution shows an exponential relation between height and time.

\begin{equation}\label{diff eq solution}
    h(t) = h_0 \: exp\left(-\frac{\rho g k t}{\mu L}\right)
\end{equation}

Linearizing and solving for the permeability $k$ yields:

\begin{equation}\label{lin.diff sol}
    k = -a\frac{\mu L}{\rho g}
\end{equation}

Were as $a$ is a linear relation coefficient defined as $a = \log\left(\frac{h(t)}{h_0}\right) /t$ which can be found by regression.

\subsection{Empirical model}




